\subsection{2.2-1}
    $n^3/1000-100n^2-100n+3=\Theta(n^3)$
\subsection{2.2-2}
    \begin{codebox}
        \Procname{$\proc{Selection-Sort}(A)$}
        \li \For $i \gets 1 \To \attrib{A}{length}-1$
            \Do
        \li     $\id{smallest} \gets i$
        \li     \For $j \gets i+1 \To \attrib{A}{length}$
                \Do
        \li         \If $A[j] < A[smallest]$
                    \Then
        \li             $\id{smallest} \gets j$
                    \End
                \End
        \li     $\mbox{swap } A[i] \mbox{ and } A[smallest]$
            \End
    \end{codebox}
    \textbf{Loop invariant:} Subarray $A[1 \twodots i-1]$ is sorted. \\
    Since after each iteration of the outer for loop, the elements in the
    subarray $A[i+1 \twodots \attrib{A}{length}]$ are all greater or equal to
    $A[i]$, which, according to the loop invariant, is the largest element of
    subarray $A[1 \twodots i]$, hence they also greater than every other
    element in that array. After run the algorithm for the first $n-1$
    elements, the last element left must be the largest element, so the array
    is ordered. \\
    Best case is when the input array is already ordered. \\
    The inner for loop affects the running time most significantly, it gives
    the time complexity $\Theta(\sum_{i=1}^{n-1})=\Theta(n^2)$ for both best
    case and worest case running time.
\subsection{2.2-3}
    Since the element being searched for is equally likely to be any element
    in the array, then the average elements being searched will be:
    $1/n\sum_{i=1}^n i=(n+1)/2$, which also gives the time complexity of
    $\Theta(n)$ \\
    Since all the elements will be scanned in the worest case, the time
    complexity is $\Theta(n)$
\subsection{2.2-4}
    Modify the algorithm to test what we known as the best case, and solve it
    specifically.
