\subsection{2-2}
%\begin{enumerate}[itemsep=0.5em, align=right, leftmargin=2.5em]
\begin{enumerate}[leftmargin=*]
	\item We need to show that the elements in $A'$ forms a permutation
		of the elements in $A$.
	\item \textbf{Loop invariant:} $A[j]=min{A[j\twodots \attrib{A}{length}]}$
		and the subarray $A[j\twodots \attrib{A}{length}]$ is
		a permutation of the elements in $A[j\twodots \attrib
		{A}{length}]$ at the time when the loop started.\\
		\textbf{Initialization:} On initialization, subarray
		$A[j\twodots \attrib{A}{length}]$ only contains one
		element, so its the smallest element in the subarray.\\
		\textbf{Maintenance:} The only possible operation to
		the subarray is exchange $A[j]$ and $A[j-1]$ when
		$A[j]<A[j-1]$. Since $A[j]$ is the smallest element
		in $A[j\twodots \attrib{A}{length}]$, if the exchange
		happens, $A[j-1]$ would become the smallest element
		in $A[j-1\twodots \attrib{A}{length}]$. Meanwhile,
		since subarray $A[j\twodots \attrib{A}{length}]$
		is a permutation of the elements in
		$A[j\twodots \attrib{A}{length}]$ at the time the
		loop started, after exchanging $A[j]$ and $A[j-1]$,
		$A[j-1\twodots \attrib{A}{length}]$ is still a
		permutation of the elements that were in $A[j-1\twodots]$
		at the time the loop started.\\
		\textbf{Termination:} Since $j$ decreases by $1$
		after each iteration, the condition that causes
		the loop terminate is $j=i$. By the state of the
		loop invariant, $A[i]=min{A[i\twodots \attrib
		{A}{length}]}$, and subarray
		$A[i\twodots \attrib{A}{length}]$ is a permutation
		of the elements that were in $A[i\twodots \attrib
		{A}{length}]$ at the time the loop started.
	\item \textbf{Loop invariant:} Subarray $A'[1\twodots i-1]$
		contains $i-1$ smallest elements come from
		array $A[1\twodots \attrib{A}{length}]$ and in
		ordered manner.\\
		\textbf{Initialization:} Piror to the first
		iteration of the outer \textbf{for} loop,
		$i=1$, so there is no element in $A'[1\twodots i-1]$,
		the loop invariant holds.\\
		\textbf{Maintenance:} Since $A'[1\twodots i-1]$
		is the $i-1$ smallest elements from
		$A[1\twodots \attrib{A}{length}]$, after the
		execution of the inner \textbf{for} loop, which
		makes $A'[i]$ the smallest element of subarray
		$A[i\twodots \attrib[A][length]]$, and the values
		in the subarray all come from $A[1\twodots
		\attrib{A}{length}]$, hence subarray
		$A'[1\twodots i]$ contains the $i$ smallest elements
		in $A[1\twodots \attrib{A}{length}]$.\\
		\textbf{Termination:} The loop terminates when
		$i$ reaches $\attrib{A}{length}$, by the state of
		the loop invariant, $A'[1\twodots \attrib{A}{length-1}]$
		contains $\attrib{A}{length-1}$ smallest elements of
		$A[1\twodots \attrib{A}{length}]$, moreover, since
		$A[\attrib{A}{length}]$ is the only element in the
		remaining subarray, therefore it is the largest element
		of $A[1\twodots \attrib{A}{length}]$. In conclusion,
		array $A'[1\twodots \attrib{A}{length}]$ is a permutation
		of $A[1\twodots \attrib{A}{length}]$ but in sorted manner.
	\item The worest case running time of the $\proc{Bubblesort}$
		is:
		\begin{displaymath}
			\sum_{i=1}^{\attrib{A}{length}-1}
			(n-i+1)=\frac{n^2}{2}
			=\Theta(n^2)
		\end{displaymath}
		It is the same as the worest case running time of
		insertion sort.
\end{enumerate}
