\subsection{2-3}
\begin{enumerate}[leftmargin=*]
	\item The \textbf{for} loop executes $n+1$ times,
		so the running time of the code is $\Theta(n)$.
	\item The naive polynomial evaluation algorithm is:\\
		\vspace{-1.5em}
		\begin{codebox}
		\Procname{$\proc{Poly-Eval}(A,x)$}
		\li	$y\gets 0$
		\li	\For $i\gets 1 \To \attrib{A}{length}$
			\Do
		\li		$y\gets x^{i-1}+A[i]$
			\End
		\end{codebox}
		Since it has to perform $i-1$ mutiplication and
		an addition at each iteration, the running time of
		$\proc{Poly-Eval}$ is:
		\begin{displaymath}
			\sum_{n=1}^{\attrib{A}{length}} i=\Theta(n^2)
		\end{displaymath}
		It is $n$ times slower than Horner's rule.
	\item \textbf{Initialization:} Piror to the first
		iteration, $y=0$, and since a summation with
		no terms equals to $0$, therefore the loop
		invariant holds.
		\textbf{Maintenance:} At the start of
		the iteration when $i$ reaches $t$,
		$y=\sum_{k=0}^{n-(t+1)}a_{k+t+1}x^k$.
		Prior to $i$ reaches $t-1$,
		\begin{align*}
			y&= a_tx^0+x\sum_{k=0}^{n-(t+1)}a_{k+t+1}x^k\\
			 &= a_tx^0+\sum_{k=1}^{n-t}a_{k+t}x^k\\
			 &= \sum_{k=0}^{n-t}a_{k+t}x^k
		\end{align*}
		which is the same as
		$\sum_{k=0}^{n-(i+1)}a_{k+i+1}x^k$,
		hence the loop invariant holds.\\
		\textbf{Termination:} $i$ reaches $-1$ when the loop terminates.
		Plug $i=-1$ into the equation, yield
		$y=\sum_{k=0}^{n}a_kx^k$.
	\item
\end{enumerate}
