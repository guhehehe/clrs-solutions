\subsection{2.4}
\begin{enumerate}[leftmargin=*]
    \item $(1,5)$, $(2,5)$, $(3,4)$, $(3,5)$, $(4,5)$.
    \item Array $[n\twodots 1]$ has the most inversions, since for each $i<j$,
        $A[i]>A[j]$. It has $\sum_{k=1}^{n-1}(n-k)=n^2/2$ inversions.
    \item The running time is in proportion to the amount of inversions in the
        array. Since in the \textbf{for} loop of lines 1-8, we set the key
        to $A[j]$ and $i=j-1$, and the \textbf{while} loop of lines 5-7 iterates
        the subarray $A[1\twodots j]$ and eliminates the inversions. In other
        words, the amount of inversions need to be eliminated contributes to
        the running time of insertion sort, the more the inversions, the longer
        the running time, vice versa.
    \item (In reference to clrs instructor's manual) \\
        To start off, we define a subroutine called $\proc{Merge-Inversions}$
        that has value $x$ in $L$ and $y$ in $R$ such that $x>y$, and,
        like the $\proc{Merge}$ procedure in merge sort, but count the amount
        of inversions at the same time. Consider an inversion $(i,j)$, and
        $x=A[i]$, $y=A[j]$, such that $x>y$, we claim that there is exactly
        one $\proc{Merge-Inversions}$ that involves $x$ and $y$. To see why,
        notice that since $L$ and $R$ are sorted subarrays, the only way that
        two elements can change their relative position to each other is for
        the greater one in $L$ and the smaller one in $R$ within the
        $\proc{Merge-Inversions}$, hence at least one $Merge-Inversions$
        involving $x$ and $y$. Moreover, observe that at the end of
        $\proc{Merge-Inversions}$, the elements in $L$ and $R$ are stored into
        another subarray either $L$ or $R$ in an ordered manner, therefore $x$
        and $y$ both appear in the same subarray in the right order, thus there
        is only one $\proc{Merge-Inversions}$ involving $x$ and $y$, our claim
        has been proven. \\
        We have proven that one inversion implies one
        $\proc{Merge-Inversions}$, now we prove that the claim still holds
        reversely. Observe that since we have a $\proc{Merge-Inversions}$,
        we have $x$ and $y$ such that $x>y$. Notice that either $L$ or $R$ are
        ordered, and $x$ in $L$ and $y$ in $R$, therefore $i<j$, thus $(i,j)$
        is an inversion. \\
        We have proven one-to-one relationship between $\proc{Merge-Inversions}$
        and inversions, now we count the inversions. Notice that if there
        exists an inversion $(i,j)$, there must be an $x$ that is greater than
        $y$ at some point of time. Moreover, since the subarray is ordered,
        all the elements with index greater than $i$ are also greater than $y$,
        hence they are all values that compose inversions with $y$. Let us
        denote $i'$ the index of the smallest element $z$ in $L$ that is
        greater than $y$, thus the number of inversions involving $y$ would be
        $n_1-i'+1$. We need to detect the first time such an $z$ and $y$ are
        exposed, and compute the number of inversions thereafter. \\
        The following pseudocode represents the idea above:
        \begin{codebox}
        \Procname{$\proc{Count-Inversions}(A,p,r)$}
        \li $\id{inversions}\gets 0$
        \li \If $p<r$
            \Then
        \li     $q\gets\lfloor(p+r)/2\rfloor$
        \li     $\id{inversions}\gets\id{inversions}
                +\proc{Count-Inversions}(A,p,q)$
        \li     $\id{inversions}\gets\id{inversions}
                +\proc{Count-Inversions}(A,q+1,r)$
        \li     $\id{inversions}\gets\id{inversions}
                +\proc{Merge-Inversions}(A,p,q,r)$
            \End
        \li \kw{return} $\id{inversions}$
        \end{codebox}
        \begin{codebox}
        \Procname{$\proc{Merge-Inversions}(A,p,q,r)$}
        \li $n_1\gets q-p+1$
        \li $n_2\gets r-q$
        \li \For $i\gets 1\To n_1$
            \Do
        \li     $L[i]\gets A[p+i-1]$
            \End
        \li \For $j\gets 1\To n_2$
            \Do
        \li     $R[j]\gets A[q+j]$
            \End
        \li $L[n_1+1]\gets \infty$
        \li $R[n_2+1]\gets \infty$
        \li $i\gets 1$
        \li $j\gets 1$
        \li $\id{inversions}\gets 0$
        \li $\id{counted}\gets \const{false}$
        \li \For $k\gets p\To r$
            \Do
        \li     \If $\id{counted}\gets \const{false}$
        %       \text{ and } R[j]<L[i]$
                \Then
        \li         $\id{inversions}\gets\id{inversions}
                    +n_1-i+1$
        \li         $\id{counted}\gets \const{true}$
                \End
        \li     \If $L[i]\le R[j]$
                \Then
        \li         $A[k]\gets L[i]$
        \li         $i\gets i+1$
        \li     \Else
        \li         $A[k]\gets R[j]$
        \li         $j\gets j+1$
        \li         $\id{counted}\gets\const{false}$
                \End
            \End
        \li \kw{return} $\id{inversions}$
        \end{codebox}
        On lines 14-16, we count the number of inversions upon each time $z$
        and $y$ are exposed, this is a linear time operation, thus the running
        time of the algorithm above is the same as merge sort, which is
        $\Theta(n\lg n)$.
\end{enumerate}
