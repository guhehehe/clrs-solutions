\subsection{2-1}
%\begin{enumerate}[itemsep=0.5em, align=right, leftmargin=2.5em]
\begin{enumerate}[leftmargin=*]
	\item Selection sort take $\Theta(k^2)$ to sort all $k$ elements in each sublist,
		hence the cost of $n/k$ sublists will be $nk$.
	\item Since we start merging when we split up to $n/k$ sublists, the depth
        of the recursion tree is $\lg(n/k)$. Moreover, we merge $n$ elements
        at each level of the tree, hence the worst case running time is
        $\Theta(n\lg(n/k))$
	\item Notice that the largest asymptotic value of $k$ is $k=\lg n$, because
		otherwise there would be higher order term in $\Theta(nk+n\lg(n/k))$
		than $\Theta(n\lg n)$. Replace $k$ with $\lg n$ in the worst case
		running time of our modified merge sort algorithm, yields
		$\Theta(n\lg n + n\lg\lg(n/k))$, which, just consider the
		higher order term, equals to the worst case running time
		of merge sort in terms of $\Theta$ notation. Therefore,
		the largest value of $k$ is $\lg n$.
	\item $k$ should be the largest sublist length so that
		selection sort outperforms merge sort.
\end{enumerate}
