\subsection{4.3-1}
    Assuming that the solution of $T(n) = T(n-1) + n$ is $O(n^2)$, so for some
    positive constant $c$ and $n \ge n_0$, we have:
    \begin{align*}
        T(n) & \le c(n - 1)^2 + n \\
             & = cn^2 - (2c + 1 )n + c \\
             & \le cn^2
    \end{align*}
    where the last step holds as long as $c \ge 1$.
\subsection{4.3-2}
    Assuming that the solution of $T(n) = T(\ceil{n/2}) + 1$ is $O(n\lg n)$,
    so for some positive constant $c$ and $n \ge n_0$, we have:
    \begin{align*}
        T(n) & \le \frac{cn\lg(n/2)}{2} + 1 \\
             & = \frac{cn(\lg n - 1)}{2} + 1 \\
             & = \frac{cn}{2}\lg n - \frac{cn+2}{2} \\
             & \le cn\lg n
    \end{align*}
    where the last step holds as long as $c \ge 2$.
\subsection{4.3-3}
    Assuming that the solution of $T(n) = 2T(\floor{n/2}) + n$ is
    $\Omega(n\lg n)$, so for some positive constant $c$ and $n \ge n_0$, we have:
    \begin{align*}
        T(n) & \ge cn\lg(n/2) + n \\
             & = cn\lg n - cn + n \\
             & = cn\lg n + n(1-c) \\
             & \ge cn\lg n
    \end{align*}
    where the last step holds as long as $c \le 1$.
\subsection{4.3-3}
    By gussing the solution to be $O(n\lg n + 1)$, we can show that $T(n) \le
    cn\lg n - cn + n + 2 \le cn\lg n + 1$ as long as $c \ge 1 + 1/n$
    ($1 + 1/n \le 2$), and the boundary condition $T(1) = 1$ holds. Hence
    the solution for the recurence in (4.19) is $O(n\lg g + 1)$, which is
    equivalent to $O(n\lg n)$.
