\subsection{3.2-1}
	$f(n)$ and $g(n)$ are monotonically increasing 
	means that if $n_1\ge n_2$, then $f(n_1)\ge
	f(n_2)$ and $g(n_1)\ge g(n_2)$. Observe that
	$f(n_1)+g(n_1)$ also greater than or equal to
	$f(n_2)+g(n_2)$, therefore $f(n)+g(n)$ is
	monotonically increasing.\\
	Also notice that since $g(n_1)\ge g(n_2)$,
	and $f(n)$ is monotonically increasing, so
	$f(g(n_1))\ge f(g(n_2))$, consequently, 
	$f(g(n))$ is also monotonically increasing.\\
	Since $f(n_1)\ge f(n_2)$, 
	also because $f(n)$ and $g(n)$ are nonnegative, 
	then $f(n_1)\cdot g(n_1)\ge f(n_2)\cdot g(n_1)$. 
	Moreover,
	$g(n_1)\ge g(n_2)$, therefore $f(n_1)\cdot g(n_1)
	\ge f(n_2)\cdot g(n_2)$, hence $f(n)\cdot g(n)$
	is monotonically increasing.
\subsection{3.2-2}
	\begin{align*}
		a^{\log_b{c}} &= a^{\frac{\log_a{c}}{\log_a{b}}}\\
					  &= a^{\log_a{c}\log_b{a}}\\
					  &= c^{\log_b{a}}
	\end{align*}
\subsection{3.2-3}
	%To prove $\lg(n!)=\Theta(n\lg n)$, we have to
	%show that there exist positive constants $c_1$,
	%$c_2$ and $n_0$ such that $0\le c_1n\lg n\le 
	%\lg(n!)\le c_2n\lg n$ for all $n\ge n_0$.\\
	%For the left part of the inequation, according
	%to Theorem 3.18 we know
	%$\lg(n!)=\lg\sqrt{2\pi n}+n\lg(n/e)+\lg(1+
	%\Theta(1/n))$, hence
	%$c_1=1/e$ satisfy the left part of the inequation
	%for all $n\ge 0$.\\
	%For the right part  of the inequation, we know
	%$c_2n\lg n=\lg n^{c_2n}$, moreover, since $\lg n$
	%is strictly increasing and $n!\le n^n$, therefore
	%$c_2=1$ can satisfy the right part of the 
	%inequation for all $n\ge 0$.\\
	Since $n!=\sqrt{2\pi}n^{n+1/2}e^{-n}$,
	\begin{align*}
		\lg (n!)	&\approx	n\lg n - n + \frac{1}{2}
		\lg(2\pi n)	\\
				&=(n+\frac{1}{2})\lg n -n + \frac
				{1}{2}\lg(2\pi)	\\
				&\approx n\lg n -n	\\
				&=	\Theta(n\lg n)
	\end{align*}
	Since $n!/n^n=(n-1)!/n^{n-1}\ldots (n-(n-1))!/
	n^{n-(n-1)}=1/n$, so $\lim_{n\to \infty}n!/n^n=0$,
	which means $n!=o(n^n)$.\\
	We can easily verify that when $n\ge4$, $0\le c2^n
	\le n!$ for any positive constant $c$, hence 
	$n!=\omega(2^n)$.
\subsection{3.2-4}
\subsection{3.2-5}
\subsection{3.2-6}
	Substitute the golden ratio and its conjugate in
	the equation, which shows they both satisfies the
	equation.
\subsection{3.2-7}
	When $i=0$, 
	\begin{align*}
		F_i &= \frac{\phi^0-\hat{\phi^0}}{\sqrt{5}}\\
			&= 0
	\end{align*}
	When $i=1$, 
	\begin{align*}
		F_i &= \frac{\phi^1-\hat{\phi}^1}{\sqrt{5}}\\
			&= 1
	\end{align*}
	Assume that when $i=k-1$,
	\begin{align*}
		F_i &= \frac{\phi^{k-1}-\hat{\phi}^{k-1}}{\sqrt{5}}
	\end{align*}
	and when $i=k$, 
	\begin{align*}
		F_i &= \frac{\phi^{k}-\hat{\phi}^k}{\sqrt{5}}
	\end{align*}
	therefore, when $i=k+1$,
	\begin{flalign*}
		F_i &= F_k+F_{k-1} \\
			&= \frac{\phi^i-\hat{\phi}^i}{\sqrt{5}} +
				\frac{\phi^{i-1}-\hat{\phi}^{i-1}}{\sqrt{5}} \\
			&= \frac{\phi^{i-1}(\phi+1)-\hat{\phi}^{i-1}(\hat{\phi}+1)}{\sqrt{5}}\\
			&= \frac{\phi^{i-1}(\phi-\phi\hat{\phi})-\hat{\phi}^{i-1}(\hat{\phi}-\phi\hat{\phi})}{\sqrt{5}}
			&&\mbox{Since }\phi\hat{\phi}=-1\\
			&= \frac{\phi^i(1-\hat{\phi})-\hat{\phi}^i(1-\phi)}{\sqrt{5}}\\
			&= \frac{\phi^{i+1}-\hat{\phi}^{i+1}}{\sqrt{5}}
			&&\mbox{Since }\phi+\hat{\phi}=1\\
	\end{flalign*}
	In conclusion, 
	\begin{align*}
		F_i &= \frac{\phi^i-\hat{\phi}^i}{\sqrt{5}}
	\end{align*}
\subsection{3.2-8}
